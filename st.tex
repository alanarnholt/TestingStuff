\documentclass{beamer}
\usetheme{Darmstadt}
\usecolortheme{seagull}
\usefonttheme[onlylarge]{structurebold}
\setbeamerfont*{frametitle}{size=\normalsize,series=\bfseries}
\setbeamertemplate{navigation symbols}{}
\newcommand{\fullpageimage}[1]{
\setbeamertemplate{background canvas}{\centering\includegraphics[width=
\paperwidth,height=\paperheight,keepaspectratio]{{#1}}}
\begin{frame}[plain]{}\end{frame}
}
 
\usepackage[utf8]{inputenc}
%\usepackage[T1]{fontenc}
\usepackage[english]{babel}
\usepackage{xspace}
\usepackage{alltt}
%\usepackage[colorlinks=true,urlcolor=blue]{hyperref}

\usepackage{Sweave}
\setkeys{Gin}{width=0.95\textwidth}

\newcommand{\code}[1]{\texttt{\small #1}}
\newcommand{\strong}[1]{{\normalfont\fontseries{b}\selectfont #1}}
\definecolor{dark-green}{rgb}{0,0.45,0}
\let\pkg=\strong
\def\RR{\textsf{R}\xspace}
\def\SP{\texttt{S-PLUS}\xspace}
\def\SS{\texttt{S}\xspace}
\def\SIII{\texttt{S3}\xspace}
\def\SIV{\texttt{S4}\xspace}

\newcommand{\email}[1]{\href{mailto:#1}{\normalfont\texttt{#1}}}

\title{\bf Analysing spatio-temporal data with R }
\author{Edzer Pebesma, Benedict Gr\"{a}ler}

\institute[University of M\"unster]
{
\includegraphics[width=1.5cm]{rdc} \hspace{.3cm}
\includegraphics[width=3.6cm]{ifgi-logo_int}\hspace{.3cm}
\includegraphics[width=3.2cm]{52n_logo+claim} \\ \vspace{.5cm}
{\color{gray}\tt edzer.pebesma@uni-muenster.de}
}
\date{\small Spatial Statistics workshop, Columbus OH\\ Jun 4, 2013}



\begin{document}
\Sconcordance{concordance:st.tex:st.Rnw:%
1 45 1 1 7 1 6 3 1 1 7 5 1 1 4 %
2 1 1 2 12 0 1 1 12 0 1 2 16 1 %
1 2 1 0 1 1 12 0 1 5 48 1 1 3 2 %
0 1 1 6 0 1 1 5 0 1 1 5 0 1 1 8 %
0 1 1 9 0 1 2 4 1 1 2 9 0 1 1 8 %
0 1 1 5 0 2 1 9 0 1 2 49 1}



\begin{frame}
  \titlepage
\end{frame}



\begin{frame}[fragile,plain]{What are spatio-temporal data?}
\begin{tiny}
\begin{Schunk}
\begin{Sinput}
> head(cars)
\end{Sinput}
\begin{Soutput}
  speed dist
1     4    2
2     4   10
3     7    4
4     7   22
5     8   16
6     9   10
\end{Soutput}
\begin{Sinput}
> summary(cars)
\end{Sinput}
\begin{Soutput}
     speed           dist       
 Min.   : 4.0   Min.   :  2.00  
 1st Qu.:12.0   1st Qu.: 26.00  
 Median :15.0   Median : 36.00  
 Mean   :15.4   Mean   : 42.98  
 3rd Qu.:19.0   3rd Qu.: 56.00  
 Max.   :25.0   Max.   :120.00  
\end{Soutput}
\end{Schunk}
\end{tiny}
\code{?cars} reveals these data were recorded in the 1920s. 
The metric units (mph, ft) suggest: UK or US. 

Are these data spatio-temporal?

\pause
No -- we (geographers, geoinformaticians, geo-whatevers)
\begin{itemize}
\item 
{\color{red} insist on {\em known} coordinates $x,y,t$}
\item prefer known reference systems -- but don't insist?
\end{itemize}
\end{frame}

\begin{frame}[fragile,plain]{Are these spatio-temporal data?}
\begin{tiny}
\begin{Schunk}
\begin{Sinput}
> data(Produc, package="plm")
> head(Produc)
\end{Sinput}
\begin{Soutput}
    state year     pcap     hwy   water    util       pc   gsp    emp unemp
1 ALABAMA 1970 15032.67 7325.80 1655.68 6051.20 35793.80 28418 1010.5   4.7
2 ALABAMA 1971 15501.94 7525.94 1721.02 6254.98 37299.91 29375 1021.9   5.2
3 ALABAMA 1972 15972.41 7765.42 1764.75 6442.23 38670.30 31303 1072.3   4.7
4 ALABAMA 1973 16406.26 7907.66 1742.41 6756.19 40084.01 33430 1135.5   3.9
5 ALABAMA 1974 16762.67 8025.52 1734.85 7002.29 42057.31 33749 1169.8   5.5
6 ALABAMA 1975 17316.26 8158.23 1752.27 7405.76 43971.71 33604 1155.4   7.7
\end{Soutput}
\end{Schunk}
\end{tiny}
Answer:
\pause
\begin{itemize}
\item {\color{red} Yes}, if you're willing to do a lot of understanding,
\item {\color{red} No}, if you don't know where ALABAMA is, or from which
country it is a state, or if you don't know what the {\em year} 1970 refers to.
\item state and year do refer, but in a soft (unstandardized) way.
\end{itemize}
\end{frame}

\frame{\frametitle{Organisation}
\begin{description}
\item[9:00–10:30] R, spatial data in R
\item[11:00-12:30] time, time series data in R
\item[14:00-15:30] spatio-temporal data types, operations, statistics
\item[16:00-17:00] ... ctd., flexible, exercises, interaction.
\end{description}
}

\begin{frame}[fragile,plain]{R}
\begin{itemize}
\item what is R?
\item what is an R package? what is a vignette?
\item what is CRAN?
\item where is the spatial task view? spatio-temporal task view?
\item how do I find in package {\em x} how to do task {\em y}?
\item how do I find out how to do ... with R?
\end{itemize} \pause
\begin{alltt}
\begin{small}
> library(fortunes)
> fortune("only how")

Evelyn Hall: I would like to know how (if) I can extract some 
of the information from the summary of my nlme.

Simon Blomberg: This is R. There is no if. Only how.
   -- Evelyn Hall and Simon 'Yoda' Blomberg
      R-help (April 2005)
\end{small}
\end{alltt}
\end{frame}

\begin{frame}[fragile,plain]{I will assume you understand this:}
\begin{columns}
\begin{column}{6cm}

\begin{tiny}
\begin{Schunk}
\begin{Sinput}
> a = data.frame(varA = c(1,1.5,2), 
+ 	varB = c("a", "a", "b"))
> a[1,]
\end{Sinput}
\begin{Soutput}
  varA varB
1    1    a
\end{Soutput}
\begin{Sinput}
> a[,1]
\end{Sinput}
\begin{Soutput}
[1] 1.0 1.5 2.0
\end{Soutput}
\begin{Sinput}
> a[[1]]
\end{Sinput}
\begin{Soutput}
[1] 1.0 1.5 2.0
\end{Soutput}
\begin{Sinput}
> a[1]
\end{Sinput}
\begin{Soutput}
  varA
1  1.0
2  1.5
3  2.0
\end{Soutput}
\begin{Sinput}
> a[, 1, drop=FALSE]
\end{Sinput}
\begin{Soutput}
  varA
1  1.0
2  1.5
3  2.0
\end{Soutput}
\end{Schunk}
\end{tiny}

\end{column}
\begin{column}{6cm}
\begin{tiny}
\begin{Schunk}
\begin{Sinput}
> a["varA"]
\end{Sinput}
\begin{Soutput}
  varA
1  1.0
2  1.5
3  2.0
\end{Soutput}
\begin{Sinput}
> a[c("varA", "varB")]
\end{Sinput}
\begin{Soutput}
  varA varB
1  1.0    a
2  1.5    a
3  2.0    b
\end{Soutput}
\begin{Sinput}
> a$varA
\end{Sinput}
\begin{Soutput}
[1] 1.0 1.5 2.0
\end{Soutput}
\begin{Sinput}
> a$varA <- 3:1
> a
\end{Sinput}
\begin{Soutput}
  varA varB
1    3    a
2    2    a
3    1    b
\end{Soutput}
\end{Schunk}
\end{tiny}
\end{column}
\end{columns}
\end{frame}

\frame{\frametitle{Spatial data}
Spatial data refresher:
\begin{itemize}
\item points, lines, polygons, grids
\item storage: shapefiles, grid files, in- or out-of-memory
\item data bases (e.g. PostGIS): geometry + attributes
\item topology representation of polygons
\item spatial indexes
\item projected data, or long/lat?
\end{itemize}
}

\frame{\frametitle{What makes a GIS a GIS?}
\pause
\begin{itemize}
\item {\color{red}store, retrieve} spatial data
\item {\color{red}visualize} spatial data
\item {\color{red}manipulate} spatial data
\item {\color{red}analyze, model} spatial data
 \begin{itemize}
 \item analyze attributes, as in a data base
 \item analyze geometries, or attributes depending on geometry
 \end{itemize}
\end{itemize}

{\em ``A geographic information system is a system designed to
capture, store, manipulate, analyze, manage, and present all types
of geographical data''} (wikipedia, from esri.com)

{\em ``In the simplest terms, GIS is the merging of cartography, statistical
analysis, and database technology.''} (wikipedia)
}

\begin{frame}[fragile,plain]{{\em How} to get spatial data into R?}
Simple answer: using \code{rgdal} (\code{readGDAL} or \code{readOGR}).
\pause
More complete:
\begin{itemize}
\item \code{readGDAL} or \code{readOGR} read the whole data set from disk 
into R, that is, into the computers main or working memory (``RAM'').
\item for grids, there are low-level routines: \code{GDAL.open} opens a
file, and \code{getRasterData} (or \code{getRasterTable}) to read {\em portions} of data; but
also (little known!):
\begin{small}
\begin{Schunk}
\begin{Sinput}
> library(rgdal)